\chapter{Background and Literature Review} \label{chap:sota}

\minitoc \mtcskip \noindent

This section aims to analyse and reflect about some works and topics that will be relevant to fully understand the problem. The study of solutions found by other authors can simplify the difficult task that is the analysis of social media data.
Hence, this section has been divided into several parts in order to perceive not only the environment in which the problem is located but also the most important points to be studied in order to build our the final product. Respectively to the problem scope its important to know what is a smart city and how the transportation system can contribute to this meaning. Since our product is a framework which goal is to extract information about social media data, i.e. texts from the Twitter, its interesting obtain the knowledge about extraction tools, such as the Twitter APIs, in order to have an idea how to construct our crawler module. The meaning of text mining and how the information present in texts can be extracted through different kinds of techniques, regarding disambiguation, filtering and modeling. Finally, it is also important to analyze several works about sentiment analysis in order to know the different methodologies and which are the most advantageous for this problem.

\section{Smart Cities and Intelligent Transportation Systems}\label{sec:smartcities}
