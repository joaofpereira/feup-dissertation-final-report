\chapter{Conclusions and Future Work} \label{chap:conclusions}

\minitoc \mtcskip \noindent

\section{Final Remarks}
The literature review, studied in Chapter~\ref{chap:sota}, shows the main challenges across the evolution process of a city in order to be titled as \textit{smart}. Moreover, the biggest restrictions in the development of intelligent systems using social media data are enunciated as well as possible methodologies and techniques to solve it. By combining the challenges of a \textit{smart city} and the restrictions present in the analysis of text messages, in particular, social media messages, the problem around this dissertation was divided into five distinct ones. The solutions presented to each of the sub-problems took into consideration lacks observed in the literature review. Domain generalization in the conception of an automatic system capable of collect, filter, processing, aggregate and demonstrate, through graphical representations, valuable information to final entities/users is one of the identified lacks, in terms of being constructed with the support of supervised methods. The transportation domain also present lacks regarding discrimination of travel-related tweets using methods that take advantages from semantic and syntactic similarities in texts. The majority of works present conventional techniques such as bag-of-words, which although good performances represents high risks when implementing supervised learning models due to the possibility of overfit the model in the training routine.

Assuming the challenges identified for this dissertation as well as the previous mentioned lacks in the literature review, we propose and develop a domain-agnostic framework and test it using five different cities over the world as use cases for three distinct analysis: simple statistics, travel-related classification for English and Portuguese languages and Twitter topics identification over two Brazilian megacities, Rio de Janeiro and São Paulo.

Travel-related classification of tweets using a combined approach of bag-of-words and bag-of-embeddings proved, as P. Saleiro~\cite{saleiro2017texrep} also reported for sentiment polarity of financial tweets, that each representation completes the other since results showed consistency and robustness over the classification performed to Portuguese texts. On the contrary, English speaking tweets do not present similar results, however, as it was previous mentioned, there was a suspicion of overffiting in the model training process. Further experiments, proved such theory since the model was able to maintain its performance only using bag-of-embeddings as training features.

Characterization of topics is a very common type of analysis in terms of information extraction from tweets. In this dissertation, we explore this analysis and implement a specific module for this task, integrating it in the final framework. Our experiment reveal promising results, however auxiliary methods would probably improved the information obtained turning it more concise and accurate. Literature review shows approaches potential which can help in this future improvements of the model responsible for identification of topics in tweets.

It is worth nothing that every experiment was performed having only into consideration geo-located tweets. This choice, as previously mentioned, was due to the additional information contained in this type of tweets. By analyse the location and combine it with results of the classification tasks, characterizations and studies over specific areas in cities are possible to be reported as weel as identifying existent and notable patterns regarding travel-related problems and urban dynamics.

Having all considered, it is necessary to divulge that the final framework is still far of its full potential, and supported on this, we consider it as a scalable, flexible and adaptive prototype that must improved over time since implementing supervised learning systems is a laborious and time-consuming process

\section{Contributions}


\section{Future Work}

