\chapter{Introduction} \label{chap:intro}

\minitoc \mtcskip \noindent

\section{Context and Motivation}\label{sec:context_motivation}

In the last few years, the rise of Web 2.0, seen as the evolution of conventional Web services into collaborative and social platforms~\cite{chi2008social}, conducted to an excessive amount of User Generated Content~\cite{kaplan2010users} (UGC) being placed \textit{online} by the population. Due to this emergency of web-content, the research community has been exploring it in order to extract added-value information regarding a large diversity of domains, such as opinion mining, human behavior and respective activity patterns, political issues, social communication (e.g. news websites). Social media platforms, more specifically, social media content (SMC), a type of UGC, has been targeted by several scientific researches focused mostly in the text mining area. Although the application of SMC in the previous mentioned domains, the \textit{smart cities}~\cite{batty2012smart} and, in particular, the transportation~\cite{gal2014potential} domain are under a smooth growth, meaning that a large path is still unexplored allowing new opportunities and challenges for the research community to reach its full potential~\cite{kn:Musto2015}.

Availability and authenticity are some of the social media content advantages considering that such information do not require additional costs regarding its exploration, is, \textit{a priori}, generated by humans, transcending a certain level of credibility and, lastly, due to the availability of tools provided by social media platforms, we can store the data and perform off-line analysis~\cite{kuflik2017automating}. Twitter is considered a MicroBlog, a type of social network, which content is similar to SMS-like messages, characteristic of a 140-characters length, and the 11th most visited website in the world~\footnote{http://www.alexa.com/siteinfo/twitter.com}. This platform has already proved its value and potential in domains ranging from news detection~\cite{kn:Sankaranarayanan2009} to real-time traffic sensing~\cite{carvalho2010real} being for this reason one of the most explored sources of data during the conduction of research studies.

Mining Twitter data is although the availability and free cost, a laborious and time-consuming process due to the restrictions and difficulties present in its content. The informal language, the existence of slang, abbreviations, jargons and the short length of the message are some of the problems when analyzing this data. Harvesting tweets automatically and, at the same time, extracting valuable information for the target domains delineated in this dissertation makes the task even more complex. However, by surpassing the previous mentioned problems, the extracted information may be of extremely importance and useful to the final stakeholders, namely \textit{smart cities} and transportation entities, during decision-making policies to improve their services.

\section{Problem Statement}\label{sec:problem}
The problem around this dissertation establishes in the analysis of a continuous flow of social media streams, in particular from Twitter, about a target scenario, as, for example, the quality of the urban transportation services in Porto. Hence, it will be necessary to filter the relevant messages, extract sentiment aspects/topics with polarity and create useful aggregated data visualizations. The problem presented can be divided in five distinct points:

\begin{enumerate}
	\item \textbf{Data collection for our target scenario}
	
	At this point, a real scenario must be chosen so that a case study can be produced.
	
	\item \textbf{Named Entity Disambiguation and content filtering}
	
	The identification of the entities present in the messages is a important point in order to disambiguate some mentions that could not be related with target scenario, making the filtering task easier, since only messages that are related must appear in the final dataset.
	
	\item \textbf{Identification of aspects/topics in Twitter messages}
	
	Since each opinion usually has a target aspect/topic about an entity, or even a service of a city, the recognition of it is relevant so that the sentiment present in the message has an orientation.
	
	\item \textbf{Sentiment polarity classification}
	
	The polarity of a sentiment may have three different types: positive, negative or neutral. At this point, it will be necessary to estimate the level of polarity expressed in the message.
	
	\item \textbf{Data aggregation and visualization}
	
	The aggregation of the results provide by all other tasks is needed. Some messages could refer to the same aspect through different ways, so it will be  necessary aggregate messages that present this characteristic. Another important task of the aggregation is the continuously calculation of the results So that when an user access the analytics UI, the results are presented immediately, without waiting time. At the visualization of the results, some qualitative and quantitative indicators may be presented to the end-user to make the analysis easier.
	
\end{enumerate}

Taking into account all the aforementioned points, the final goal of this dissertation is to create a framework, based on the concept of analysis. The framework should be capable of automatically processing social media texts, regarding semantic processing, topic detection and sentiment analysis. An user interface will be provide to the end-user, illustrating a set of qualitative and quantitative indicators to analysis. This knowledge can be relevant both to users of a particular service or to the responsible entities in order to improve the decision-making process.

\section{Goals and Expected Contributions}\label{sec:contributions}


\section{Dissertation Structure}\label{sec:diss_structure}
The effort applied to this dissertation generated a great diversity of points and due to that the remaining document is structured into four chapters.
Section \ref{chap:sota} starts with a brief conceptualization in the Smart Cities and Intelligent Transportation System domains, as well as previous related works using social media content as its basis.
The proposed framework is referenced in Section \ref{chap:framework}, being each its composing modules depth described.
Experiments performed to test each module of the framework are reported in Section~\ref{chap:experiments}.
Completing this document, conclusions, future work and a few final remarks are exposed in chapter~\ref{chap:conclusions}.