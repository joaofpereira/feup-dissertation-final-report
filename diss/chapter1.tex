\chapter{Introduction} \label{chap:intro}

\minitoc \mtcskip \noindent

\section{Context and Motivation}\label{sec:context_motivation}

In the last few years, the rise of Web 2.0, seen as the evolution of conventional Web services into collaborative and social platforms~\cite{chi2008social}, conducted to an excessive amount of User Generated Content~\cite{kaplan2010users} (UGC) being placed \textit{online} by the population. Due to this emergency of web-content, the research community has been exploring it in order to extract added-value information regarding a large diversity of domains, such as opinion mining, human behavior and respective activity patterns, political issues, social communication (e.g. news websites). Social media platforms, more specifically, social media content (SMC), a type of UGC, has been targeted by several scientific researches focused mostly in the text mining area. Although the application of SMC in the previous mentioned domains, the \textit{smart cities}~\cite{batty2012smart} and, in particular, the transportation~\cite{gal2014potential} domain are under a smooth growth, meaning that a large path is still unexplored allowing new opportunities and challenges for the research community to reach its full potential~\cite{kn:Musto2015}.

Availability and authenticity are some of the social media content advantages considering that such information do not require additional costs regarding its exploration, is, \textit{a priori}, generated by humans, transcending a certain level of credibility and, lastly, due to the availability of tools provided by social media platforms, we can store the data and perform off-line analysis~\cite{kuflik2017automating}. Twitter is considered a MicroBlog, a type of social network, which content is similar to SMS-like messages, characteristic of a 140-characters length, and the 11th most visited website in the world~\footnote{http://www.alexa.com/siteinfo/twitter.com}. This platform has already proved its value and potential in domains ranging from news detection~\cite{kn:Sankaranarayanan2009} to real-time traffic sensing~\cite{carvalho2010real} being for this reason one of the most explored sources of data during the conduction of research studies.

Mining Twitter data is although the availability and free cost, a laborious and time-consuming process due to the restrictions and difficulties present in its content. The informal language, the existence of slang, abbreviations, jargons and the short length of the message are some of the problems when analyzing this data. Harvesting tweets automatically and, at the same time, extracting valuable information for the target domains delineated in this dissertation makes the task even more complex. However, by surpassing the previous mentioned problems, the extracted information may be of extremely importance and useful to the final stakeholders, namely \textit{smart cities} and transportation entities, during decision-making policies to improve their services.

\section{Problem Statement}\label{sec:problem}
The problem around this dissertation is focused in the analysis of a continuous flow of social media streams provided by Twitter. To analyse such streams, multiple steps composed in an iterative process are needed in order to filter out non-related content and proceed with extraction of information about a specific scenario. Here, since the target scenarios are associated to \textit{smart cities} and transportation domains, data related to it must be explored and analysed. To the best of our knowledge, there are no public datasets related to these domains and the creation of a gold standard dataset constitutes a complex endeavor, which is, for this reason, an obstacle to surpass in this dissertation. The extraction of information from social media content is another overwhelming task since it is necessary the application of several NLP methods in order to minimize/extinct its peculiarly problems. Hence, the main problem can be divided in five distinct sub-problems:

\begin{enumerate}
	\item \textbf{Data collection method for various locations}\\
	Choosing a method to collect data that provides a large range of valuable information for different cities constitutes the first sub-problem.
	
	\item \textbf{Content filtering}\\
	It is necessary the guarantee of that all information is fully related to the target scenario in the analysis, as well as removing messages which does not brought additional information (for instance, tweets only composed by \textit{emoticons}) or are not related to the end-users expectations, i.e. if we are targeting content from a specific city, we must assure that such content is indeed posted when users were there.
	
	\item \textbf{Identification of topics in Twitter messages}\\
	The identification of topics in Twitter messages is a very important point in the analyses of the \textit{smart cities} context. This task allows the identification of what is been talked about recently and also where the conversation topics are geographically distributed.
	
	\item \textbf{Travel-related classification}\\
	In order to produce valuable information for the transportation services, we need to analyse the content of a message and verify if it is truly related with the domain in study. Hence, discriminate travel-related tweets is one of the sub-problems that must be tackled.
	
	\item \textbf{Data aggregation and visualization}\\
	The aggregation of the results provide by all other tasks is needed. This aggregation task may be continuously calculating the results in order to make the user experience easier and smooth without taking to much response time by the data visualization UI. The graphical visualizations should be of easy interpretation by the end-user and having this in mind some qualitative and quantitative indicators may be presented.
\end{enumerate}

\section{Goals and Expected Contributions}\label{sec:contributions}
Following the previous mentioned problem in Section \ref{sec:problem}, the main goal of this dissertation passes through the development of a prototype framework based on the concept of analysis. Such framework demands a solution for each of the aforementioned sub-problems, and for that reason modularity is needed in the design and implementation of the final tool. Its usability will be directed to companies or even ordinary users and should be able to provide relevant information about a specific real-world scenario under the \textit{smart cities} and transportation fields. The framework should be capable of automatically processing social media texts, more specifically, general topic detection and characterization of travel-related tweets. The following list summarizes the crucial goals behind this dissertation:

\begin{itemize}
	\item Extraction of valuable information from Social Media Content to the Transportation and \textit{Smart Cities} domains;
	\item Designing and implementation of a framework capable of automatize the analysis process;
	\item Application, when possible, of recent advances in the area of text analysis;
\end{itemize}

\medskip

In terms of expected contributions, we hope that such generated information through the framework data analytics may be relevant both to ordinary users of a particular service and to the responsible entities in order to improve decision-making policies.

\section{Publications}\label{sec:publications}
In this section several scientific contributions performed during the period of this dissertation are mentioned:

\begin{itemize}
		\item
		João Pereira, Arian Pasquali, Pedro Saleiro and Rosaldo J. F. Rossetti. {\color{blue}Transportation in Social Media: an automatic classifier for travel-related tweets}. In \emph{Portuguese Conference on Artificial Intelligence} (EPIA), 2017. In Press.
		
		\item
		João Pereira, Arian Pasquali, Pedro Saleiro and Rosaldo J. F. Rossetti. {\color{blue}Classifying Travel-related Tweets using Word Embeddings}. In \emph{International Conference on Information and Knowledge Management} (CIKM), 2017. Under review.
		
		\item
		João Pereira, Arian Pasquali, Pedro Saleiro and Rosaldo J. F. Rossetti. {\color{blue}Characterizing Geo-located Tweets in Brazilian Megacities}. In \emph{IEEE International Summer School on Smart Cities} (IEEE S3C), 2017. Under review.
\end{itemize}

\section{Dissertation Structure}\label{sec:diss_structure}
The effort applied to this dissertation generated a great diversity of points and due to that the remainder of the document is structured into four chapters.
Section \ref{chap:sota} starts with a brief conceptualization in the Smart Cities and Intelligent Transportation System domains, as well as previous related works using social media content as its basis.
The proposed framework is referenced in Section \ref{chap:framework}, being each its composing modules depth described.
Experiments performed to test each module of the framework are reported in Section~\ref{chap:experiments}.
Completing this document, conclusions, future work and a few final remarks are exposed in chapter~\ref{chap:conclusions}.