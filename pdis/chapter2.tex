\chapter{Revisão Bibliográfica} \label{chap:sota}

\section*{}

Neste capítulo é descrito o estado da arte e são
apresentados trabalhos relacionados para mostrar o que existe no
mesmo domínio e quais os problemas em aberto.
Deve deixar claro que existe uma oportunidade de desenvolvimento que
cobre alguma falha concreta .

O capítulo deve também efetuar uma revisão tecnológica às principais
ferramentas utilizáveis no âmbito do projeto, justificando futuras
escolhas.

\section{Introdução}

Neste capítulo é ilustrada a utilização de macros \LaTeX\ para definir
entradas no índice remissivo e são feitas diversas referências
bibliográficas, usando-se texto de um artigo apresentado na Conferência 
XATA2006~\cite{kn:MVL06-xata}.

Nos últimos tempos têm surgido diversas soluções, apresentadas por
empresas do sector Automação de Sistemas para a disponibilização de
sistemas \scadadms{} na \textit{Web}.

\section{Trabalhos Relacionados}\label{sec:relatedWork}

\emph{Scalable Vector Graphics}\index{SVG}\index{XML!SVG} é uma
linguagem em formato XML que descreve gráficos de duas dimensões. 
Este formato padronizado pela W3C (\emph{World Wide Web Consortium})
é livre de patentes ou direitos de autor e está totalmente
documentado, à semelhança de outros W3C
standards~\cite{kn:svgdoc}.

Sendo uma linguagem XML, o \svg{} herda uma série de vantagens: a
possibilidade de transformar \svg{} usando técnicas como
XSLT\index{XML!XSLT}, de embeber \svg{} em qualquer documento
XML\index{XML} usando \textit{namespaces} ou até de  
estilizar \svg{} recorrendo a CSS\index{CSS} (\emph{Cascade Style Sheets}). 
De uma forma geral, pode dizer-se que \svg{}s interagem bem com as
atuais tecnologias ligadas ao XML e à Web, tal como referido
em~\cite{kn:svgibm,kn:svgw3c}.

Quisque tristique, metus eu iaculis
sagittis, urna leo bibendum diam, a ultricies sem diam a augue. Mauris
consectetuer, libero vel euismod tincidunt, nisi metus viverra ante,
quis pretium sapien odio nec risus. Nunc semper auctor
nulla\footnote{Exemplo de nota de rodapé.}.

\section{Tecnologias Existentes} \label{sec:relatedTech}

\section{Resumo ou Conclusões}

No final do capítulo deverá ser apresentado um resumo com as 
principais conclusões que se podem tirar. 

Vivamus non nunc nec risus tempor varius. Quisque bibendum mi at
dolor. Aliquam consectetuer condimentum risus. Aliquam luctus pulvinar
sem. Duis aliquam, urna et vulputate tristique, dui elit aliquet nibh,
vel dignissim magna turpis id sapien. Duis commodo sem id
quam. Phasellus dolor. Class aptent taciti sociosqu ad litora torquent
per conubia nostra, per inceptos himenaeos. 
