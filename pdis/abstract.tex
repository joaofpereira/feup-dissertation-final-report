\chapter*{Resumo}

O Resumo fornece ao leitor um sumário do conteúdo da dissertação.
Deverá ser breve mas conter detalhe suficiente e, uma vez que é a porta
de entrada para a dissertação, deverá dar ao leitor uma boa impressão
inicial.

Este texto inicial da dissertação é escrito no fim e resume numa
página, sem referências externas, o tema e o contexto do trabalho, a
motivação e os objectivos, as metodologias e técnicas empregues, os
principais resultados alcançados e as conclusões.

Este documento ilustra o formato a usar em dissertações na \Feup.
São dados exemplos de margens, cabeçalhos, títulos, paginação, estilos
de índices, etc. 
São ainda dados exemplos de formatação de citações, figuras e tabelas,
equações, referências cruzadas, lista de referências e índices.
%Este documento não pretende exemplificar conteúdos a usar. 
É usado texto descartável, \emph{Loren Ipsum}, para preencher a
dissertação por forma a ilustrar os formatos.

Seguem-se umas notas breves mas muito importantes sobre a versão 
provisória e a versão final do documento. 
A versão provisória, depois de verificada pelo orientador e de 
corrigida em contexto pelo autor, deve ser publicada na página 
pessoal de cada estudante/dissertação, juntamente com os dois 
resumos, em português e em inglês; deve manter a marca da água, 
assim como a numeração de linhas conforme aqui se demonstra.

A versão definitiva, a produzir somente após a defesa, em versão 
impressa (dois exemplares com capas próprias FEUP) e em versão 
eletrónica (6 CDs com "rodela" própria FEUP), deve ser limpa da marca de 
água e da numeração de linhas e deve conter a identificação, na primeira 
página, dos elementos do júri respetivo. 
Deve ainda, se for o caso, ser corrigida de acordo com as instruções 
recebidas dos elementos júri.

Lorem ipsum dolor sit amet, consectetuer adipiscing elit. Sed vehicula
lorem commodo dui. Fusce mollis feugiat elit. Cum sociis natoque
penatibus et magnis dis parturient montes, nascetur ridiculus
mus. Donec eu quam. Aenean consectetuer odio quis nisi. Fusce molestie
metus sed neque. Praesent nulla. Donec quis urna. Pellentesque
hendrerit vulputate nunc. Donec id eros et leo ullamcorper
placerat. Curabitur aliquam tellus et diam. 

Ut tortor. Morbi eget elit. Maecenas nec risus. Sed ultricies. Sed
scelerisque libero faucibus sem. Nullam molestie leo quis
tellus. Donec ipsum. Nulla lobortis purus pharetra turpis. Nulla
laoreet, arcu nec hendrerit vulputate, tortor elit eleifend turpis, et
aliquam leo metus in dolor. Praesent sed nulla. Mauris ac augue. Cras
ac orci. Etiam sed urna eget nulla sodales venenatis. Donec faucibus
ante eget dui. Nam magna. Suspendisse sollicitudin est et mi. 

Phasellus ullamcorper justo id risus. Nunc in leo. Mauris auctor
lectus vitae est lacinia egestas. Nulla faucibus erat sit amet lectus
varius semper. Praesent ultrices vehicula orci. Nam at metus. Aenean
eget lorem nec purus feugiat molestie. Phasellus fringilla nulla ac
risus. Aliquam elementum aliquam velit. Aenean nunc odio, lobortis id,
dictum et, rutrum ac, ipsum. 

Ut tortor. Morbi eget elit. Maecenas nec risus. Sed ultricies. Sed
scelerisque libero faucibus sem. Nullam molestie leo quis
tellus. Donec ipsum. 

\chapter*{Abstract}

The \textit{Smart Cities} are a future that most cities want to achieve by appealing to the citizens' participation to improve its services. The participation process is comprised of simple contributions, opinions or even validations of technological projects by the citizens. In this way, the social networks are a rich source of data where ordinary citizens publicly share their comments, and for this reason, it's resorted by many smart cities' projects.

The problem present in this dissertation is to extract information from social media texts, in this case the \textit{Twitter}, regarding the services of a \textit{smart city}. These texts that are  obtained from a social network can sometimes be difficult to analyze, since the writing of a person is not always correct and/or coherent, such as, wrongly typing words and/or incorrect phrasal structuring, respectively. With all of this points, it's possible to slice the main problem into four sub-problems. The first one is to collect data from the social network; the second one is focused on the semantic analysis and elimination of ambiguous lexicons of the text; the third one will be the aspect-based sentiment analysis present in each piece of the text; and, finally, it's necessary to aggregate the results obtained into a set of indicators to categorize the founded problems.

We propose to tackle the aforementioned problems through the creation of a framework capable of collecting and extracting value from social media texts. The framework will be composed by a total of four modules, in order to solve each sub-problem mentioned above. The first one should handle social-data-collection tasks according to some user-defined heuristics. The second module should semantically analyze the text and remove ambiguities from the lexicon, using named entity linking techniques in order to filter the noise of the data set. After this filtering, the texts will be subjected to the module of sentiment analysis to find out if its content has positive or negative value. Finally, the analytics module will aim the aggregation of the results obtained from the previously module in a set of indicators.

The expected results in this dissertation are presented in the analytics module of the framework, since the produced indicators may be a good use in decision-making process of this cities, helping in improving the citizens life quality.
